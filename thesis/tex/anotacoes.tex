%% Template para dissertacao/tese na classe UFBAthesis
%% versao 1.0
%% (c) 2005 Paulo G. S. Fonseca
%% (c) 2012 Antonio Terceiro
%% (c) 2014 Christina von Flach
%% www.dcc.ufba.br/~flach/ufbathesis

%% Carrega a classe ufbathesis
%% Opcoes: * Idiomas
%%           pt   - portugues (padrao)
%%           en   - ingles
%%         * Tipo do Texto
%%           bsc  - para monografias de graduacao
%%           msc  - para dissertacoes de mestrado (padrao)
%%           qual - exame de qualificacao de mestrado
%%           prop - exame de qualificacao de doutorado
%%           phd  - para teses de doutorado
%%         * Media
%%           scr  - para versao eletronica (PDF) / consulte o guia do usuario
%%         * Estilo
%%           classic - estilo original a la TAOCP (deprecated) - apesar de deprecated, manter esse.
%%           std     - novo estilo a la CUP (padrao)
%%         * Paginacao
%%           oneside - para impressao em face unica
%%           twoside - para impressao em frente e verso (padrao)

% Aten��o: Manter 'classic' na declaracao abaixo:
\documentclass[qual, classic, a4paper]{ufbathesis}

%% Preambulo:
\usepackage[utf8]{inputenc}
\usepackage{graphicx}
\usepackage{lipsum}
\usepackage{hyphenat}
\usepackage[usenames, dvipsnames, table]{xcolor}
\usepackage{booktabs}
\usepackage{pifont}
\usepackage{multirow}
\usepackage{listings} 
\usepackage{colortbl}
\usepackage{xfrac}
\usepackage[FIGTOPCAP]{subfigure}
\usepackage[printonlyused, withpage]{acronym}

% Universidade
\university{Universidade Federal da Bahia}

% Endereco (cidade)
\address{Salvador}

% Instituto ou Centro Academico
\institute{Instituto de Matem\'{a}tica}

% Nome da biblioteca - usado na ficha catalografica
\library{Biblioteca Reitor Mac\^{e}do Costa}

% Programa de pos-graduacao
\program{Programa de P\'{o}s-Gradua\c{c}\~{a}o em Ci\^{e}ncia da Computa\c{c}\~{a}o}

% Area de titulacao
\majorfield{Ci\^{e}ncia da Computa\c{c}\~{a}o}

% Titulo da dissertacao
\title{Anotações}

% Data da defesa
% e.g. \date{19 de fevereiro de 2013}
\date{-}
% e.g. \defenseyear{2013}
\defenseyear{-}

% Autor
% e.g. \author{Jose da Silva}
\author{Ruivaldo Azevedo Lobão Neto}

% Orientador(a)
% Opcao: [f] - para orientador do sexo feminino
% e.g. \adviser[f]{Profa. Dra. Maria Santos}
\adviser{Ricardo Araujo Rios}

%% Inicio do documento
\begin{document}

\pgcompfrontpage

%% Parte pre-textual
\frontmatter

\pgcomppresentationpage


%%%%%%%%%%%%%%%%%%%%%
% Resumo em Portugues
%%%%%%%%%%%%%%%%%%%%%

\resumo
Anotações realizadas durante a pesquisa pré-qualificação.

% Palavras-chave do resumo em Portugues
\begin{keywords}
Concept drift, change point
\end{keywords}

%%%%%%%%%%%%%%%%%%%
% Sumario / Indice
%%%%%%%%%%%%%%%%%%%

%% Parte textual
\mainmatter

% Eh aconselhavel criar cada capitulo em um arquivo separado, digamos
% "capitulo1.tex", "capitulo2.tex", ... "capituloN.tex" e depois
% inclui-los com:
% \include{capitulo1}
% \include{capitulo2}
% ...
% \include{capituloN}
%
% Importante: 
% Use \xchapter{}{} ao inves de \chapter{}; se n�o quiser colocar texto antes do inicio do capitulo, use \xchapter{texto}{}.

\xchapter{Estudos Preliminares}{}

\section{Implementação Pettitt - MOA}

\subsection{Método de Pettitt}

O método de Pettitt

%% Parte pos-textual
\backmatter

% Bibliografia
% � aconselh�vel utilizar o BibTeX a partir de um arquivo, digamos "biblio.bib".
% Para ajuda na cria��o do arquivo .bib e utiliza��o do BibTeX, recorra ao
% BibTeXpress em www.cin.ufpe.br/~paguso/bibtexpress
\bibliographystyle{abntex2-alf}
\bibliography{biblio}

%% Fim do documento
\end{document}
%------------------------------------------------------------------------------------------%
