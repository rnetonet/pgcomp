%% Template para dissertacao/tese na classe UFBAthesis
%% versao 1.0
%% (c) 2005 Paulo G. S. Fonseca
%% (c) 2012 Antonio Terceiro
%% (c) 2014 Christina von Flach
%% www.dcc.ufba.br/~flach/ufbathesis

%% Carrega a classe ufbathesis
%% Opcoes: * Idiomas
%%           pt   - portugues (padrao)
%%           en   - ingles
%%         * Tipo do Texto
%%           bsc  - para monografias de graduacao
%%           msc  - para dissertacoes de mestrado (padrao)
%%           qual - exame de qualificacao de mestrado
%%           prop - exame de qualificacao de doutorado
%%           phd  - para teses de doutorado
%%         * Media
%%           scr  - para versao eletronica (PDF) / consulte o guia do usuario
%%         * Estilo
%%           classic - estilo original a la TAOCP (deprecated) - apesar de deprecated, manter esse.
%%           std     - novo estilo a la CUP (padrao)
%%         * Paginacao
%%           oneside - para impressao em face unica
%%           twoside - para impressao em frente e verso (padrao)

% Aten��o: Manter 'classic' na declaracao abaixo:
\documentclass[qual, classic, a4paper]{ufbathesis}

%% Preambulo:
\usepackage[utf8]{inputenc}
\usepackage{graphicx}
\usepackage{lipsum}
\usepackage{hyphenat}
\usepackage[usenames, dvipsnames, table]{xcolor}
\usepackage{booktabs}
\usepackage{pifont}
\usepackage{multirow}
\usepackage{listings} 
\usepackage{colortbl}
\usepackage{xfrac}
\usepackage[FIGTOPCAP]{subfigure}
\usepackage[printonlyused, withpage]{acronym}

% Universidade
\university{Universidade Federal da Bahia}

% Endereco (cidade)
\address{Salvador}

% Instituto ou Centro Academico
\institute{Instituto de Matem\'{a}tica}

% Nome da biblioteca - usado na ficha catalografica
\library{Biblioteca Reitor Mac\^{e}do Costa}

% Programa de pos-graduacao
\program{Programa de P\'{o}s-Gradua\c{c}\~{a}o em Ci\^{e}ncia da Computa\c{c}\~{a}o}

% Area de titulacao
\majorfield{Ci\^{e}ncia da Computa\c{c}\~{a}o}

% Titulo da dissertacao
\title{Detecção de mudanças de conceito em fluxos de dados não estacionários}

% Data da defesa
% e.g. \date{19 de fevereiro de 2013}
\date{19 de Julho de 2018}
% e.g. \defenseyear{2013}
\defenseyear{2018}

% Autor
% e.g. \author{Jose da Silva}
\author{Ruivaldo Azevedo Lobão Neto}

% Orientador(a)
% Opcao: [f] - para orientador do sexo feminino
% e.g. \adviser[f]{Profa. Dra. Maria Santos}
\adviser{Ricardo Araújo Rios}

% Orientador(a)
% Opcao: [f] - para orientador do sexo feminino
% e.g. \coadviser{Prof. Dr. Pedro Pedreira}
% Comente se nao ha co-orientador
%\coadviser{Nome Completo do CO-ORIENTADOR}

%% Inicio do documento
\begin{document}

\pgcompfrontpage

%% Parte pre-textual
\frontmatter

\pgcomppresentationpage


%%%%%%%%%%%%%%%%%%%%%
% Resumo em Portugues
%%%%%%%%%%%%%%%%%%%%%

\resumo
O aprendizado a partir de fluxos de dados (aprendizagem incremental) tem crescido como foco de pesquisa, graças a existência de problemas práticos e desafios em aberto.
Dentre estes, está a detecção de mudanças de conceito, fenômeno que ocorre quando a distribução dos dados é alterada, tornando o modelo vigente impreciso ou obsoleto.
Neste trabalho, propomos uma nova técnica para detecção de mudanças de conceito.

% Palavras-chave do resumo em Portugues
\begin{keywords}
Mudança de conceito, detecção de mudanças, aprendizagem adaptativa, fluxos de dados.
\end{keywords}

%%%%%%%%%%%%%%%%%%%
% Resumo em Ingles
%%%%%%%%%%%%%%%%%%%

\abstract
Learning from data streams (incremental learning) is increasing as a research focus, due to the existence of practical problems and open challenges. Among which, is the detection of concept drift, a phenomenon that happens when the data distribution is altered, making the model inaccurate or obsolete. In this work, we propose a novel technic to detect concept drifts.

% Palavras-chave do resumo em Ingles
\begin{keywords}
Concept drift, change detection, adaptive learning, data streams.
\end{keywords}

%%%%%%%%%%%%%%%%%%%
% Sumario / Indice
%%%%%%%%%%%%%%%%%%%

% Comente para ocultar
\tableofcontents

% Lista de figuras
% Comente para ocultar
%\listoffigures

% Lista de tabelas
% Comente para ocultar
%\listoftables

%\chapter*{Lista de Siglas}

% Sintaxe da lista de acordo com a documentação do pacote `acronym'
% documentação: http://mirror.unl.edu/ctan/macros/latex/contrib/acronym/acronym.pdf
%\begin{acronym}[PGCOMP]
%    \acro{PGCOMP}{Programa de Pós-Graduação em Ciência da Computação}
    %\acro{CNPq}{Conselho Nacional de Desenvolvimento Científico e Tecnológico}
%\end{acronym}

%% Parte textual
\mainmatter

% Eh aconselhavel criar cada capitulo em um arquivo separado, digamos
% "capitulo1.tex", "capitulo2.tex", ... "capituloN.tex" e depois
% inclui-los com:
% \include{capitulo1}
% \include{capitulo2}
% ...
% \include{capituloN}
%
% Importante: 
% Use \xchapter{}{} ao inves de \chapter{}; se n�o quiser colocar texto antes do inicio do capitulo, use \xchapter{texto}{}.

% \xchapter{Introdu\c{c}\~{a}o}{Este eh o primeiro cap\'{\i}tulo, onde eu conto toda a historia deste trabalho, o problema, a solu\c{c}\~{a}o, etc.}

% É recomendável utilizar `\acresetall' no início de cada capítulo para reiníciar o contator de referências às siglas.
% \acresetall 


%\section{Se\c{c}\~{a}o}
%Trabalho do  \ac{PGCOMP}. Bolsa do \ac{CNPq}.

%\begin{figure}[h]
%Figure
%\caption{As siglas também funcionam nas legendas, seja na forma de sigla \ac{CNPq}, seja na forma completa \acf{PGCOMP}.}
%\end{figure}

%\lipsum

%\subsection{Uma Subse\c{c}\~{a}o}
%\acresetall
%Texto para mostrar como o \verb|\acresetall| funciona \ac{CNPq}, \ac{PGCOMP}. Ele reseta os contadoes e faz a sigla %aparecer na forma estendida novamente.

%\subsection{Outra Subse\c{c}\~{a}o}

%Texto  \acf{CNPq}, \acf{PGCOMP}.

\xchapter{Revis\~{a}o Bibliogr\'{a}fica}{}

\subsection{Introdução}

A extração de informações úteis a partir de grandes conjuntos de dados é uma tarefa desafiadora para os pesquisadores.
Os algoritmos de aprendizagem de máquina baseados em fluxos de dados contínuos (FCDs) atuam em um contexto diferente dos algoritmos tradicionais, 
devido a natureza dinâmica das FCDs.
Esses algoritmos devem se adaptar às constantes mudanças de distribuição dos dados, para não se tornarem imprecisos ou obsoletos.

Portanto, a atividade de Detecção de Novidades (DN) - \textit{Concept Drift} - é essencial para o bom funcionamento dessas técnicas. A atividade de DN permite identificar o surgimento de novos conceitos e mudanças em conceitos existentes, permitindo a atualização do modelo de decisão. Novas técnicas de aprendizado ativo têm sido exploradas com o objetivo de aprimorar o processo de classificação e identificação de mudanças de conceito. 

% Dessa forma, a tarefa de Detecção de Novidade (DN) é imprescindível neste pro-
% cesso, pois é ela que identiĄca o surgimento de novos conceitos e a mudança nos conceitos
% existentes, que posteriormente irão atualizar o modelo de decisão. Visando melhorar o
% processo de classiĄcação e DN em FCDs, estão sendo exploradas técnicas de aprendizado
% ativo que auxiliam no processo de escolha das instâncias a serem rotuladas para atualizar
% o modelo de decisão.

% Este capítulo descreve as diferentes situações que estão presentes em FCDs. A
% seção 2.2 apresenta os FCDs, exemplos de aplicações do mundo real que geram dados
% em Ćuxo e quais os desaĄos encontrados para manipular grandes quantidades de dados.
% A seção 2.3 traz o conceito de classiĄcação em FCDs e como os algoritmos da literatura
% foram desenvolvidos para esta Ąnalidade. A seção 2.4 detalha em qual etapa do processo
% de classiĄcação a DN é aplicada e como é realizada a atualização do modelo de decisão.
% A seção 2.5 relata como são implementadas as técnicas de aprendizado e seus desaĄos.
% A seção 2.6 faz uma breve descrição de algoritmos presentes na literatura que foram
% desenvolvidos para classiĄcar dados e detectar novidades em FCDS.


%Livro \cite{demeyer2008} e  livro \cite{raymond1999}.

%\xchapter{Exemplos}{} %sem preambulo

%A numera\c{c}\~{a}o de figuras \'{e} sequencial, dentro do cap\'{\i}tulo. Ver Figura \ref{default-regular1} e Figura %\ref{default-regular2}.

%A numera\c{c}\~{a}o de tabelaas \'{e} sequencial, dentro do cap\'{\i}tulo. Ver Tabela \ref{default-table1} e Tabela %\ref{default-table2}.


%hline
%\end{tabular}
%\end{center}
%\label{default-table2}
%\end{table}%

%\xchapter{Outro cap\'{\i}tulo}{} %sem preambulo
%\lipsum


%% Parte pos-textual
\backmatter

% Bibliografia
% � aconselh�vel utilizar o BibTeX a partir de um arquivo, digamos "biblio.bib".
% Para ajuda na cria��o do arquivo .bib e utiliza��o do BibTeX, recorra ao
% BibTeXpress em www.cin.ufpe.br/~paguso/bibtexpress
\bibliographystyle{abntex2-alf}
\bibliography{biblio}

% Apendices
% Comente se naoo houver apendices
%\appendix

%\xchapter{Exemplo de Ap\^endice}{} %sem preambulo
%\lipsum
% Eh aconselhavel criar cada apendice em um arquivo separado, digamos
% "apendice1.tex", "apendice.tex", ... "apendiceM.tex" e depois
% inclui--los com:
% \include{apendice1}
% \include{apendice2}
% ...
% \include{apendiceM}

%% Fim do documento
\end{document}
%------------------------------------------------------------------------------------------%
